\section{Task 1}
\begin{framed}\em You should hand in your domain and problem definition files. The files should be well commented: Explain the way you represent the domain and motivate your choice of predicates, objects and operators.\em
\end{framed}
Our domain consists of six types (\verb=shakey=, \verb=box=, \verb=lightswitch=, \verb=door=, \verb=toy=, \verb=room=). Shakey can hold two toys (\verb=hold_l, hold_r=) and we use a "boolean" to mark each \verb=hold_= as busy if they are. We also need to keep track of where (\verb=belong, box\_in, at, in=) things (\verb=lightswitch=, \verb=box=, \verb=shakey=, \verb=toy=) are located. At last, we need to connect the rooms, which is done through \verb=connecting= and (where applicable) labeling the door as \verb=wide=.

Motivating choice of objects? Well, they are chosen to fit the problems we wish to solve and annotated accordingly in each problem file.

\section{Task 2}
\begin{framed}\em You should hand in a written description of your experimental setup (the problem parameters you chose, how you varied them etc.) and the results of the experiment, in the form of tables and/or graphs.

If you're able, try to explain why changes in different problem parameters have (or don't have) different effects on different planners.\em\end{framed}

%%% Insert Johan's stuff here ...
